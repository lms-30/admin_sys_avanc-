\documentclass[12pt,a4paper]{report}
\usepackage[utf8]{inputenc}
\usepackage[T1]{fontenc}
\usepackage[french]{babel}
\usepackage{graphicx}
\usepackage{array}
\usepackage{hyperref}
\usepackage{caption}
\usepackage{geometry}
\geometry{margin=2.5cm}
\usepackage{longtable}
\usepackage{fancyhdr}
\usepackage{titlesec}

\titleformat{\chapter}[display]{\normalfont\bfseries\huge}{\chaptertitlename\ \thechapter}{20pt}{\Huge}

\pagestyle{fancy}
\fancyhf{}
\rhead{LAHINIRIKO}
\lhead{Mini-mémoire IoT}
\rfoot{\thepage}

\begin{document}

% Page de garde
\begin{titlepage}
    \centering
    \includegraphics[width=0.3\textwidth]{captures/logo.jpg}\par\vspace{1cm}
    {\scshape\LARGE Projet de Réseau Informatique \par}
    \vspace{1.5cm}
    {\Huge\bfseries Système de Vidéosurveillance avec IoT\par}
    \vspace{2cm}
    {\Large\itshape Réalisé par : LAHINIRIKO\par}
    \vfill
    {\large Juin 2025\par}
\end{titlepage}

\tableofcontents
\newpage

\chapter*{Introduction}
Dans ce chapitre, nous allons simuler un mini-réseau LAN avec une technologie IoT (Internet Of Things), suivi d'une simulation de l’utilisation de serveurs DNS, FTP, HTTP/HTTPS, Mails, et aussi serveur IoT pour assurer l'utilisation des objets connectés.

\chapter{Topologie}
Dans cette simulation, nous allons utiliser une topologie en étoile avec deux bâtiments :

\begin{itemize}
  \item \textbf{Bâtiment A} : contient 5 salles équipées de vidéosurveillances pour renforcer la sécurité matérielle du réseau.
  \item \textbf{Bâtiment B} : contient 2 salles : une salle de contrôle des entrées/sorties et des failles, et une salle de direction.
\end{itemize}

\begin{figure}[h!]
    \centering
    \includegraphics[width=0.8\textwidth]{captures/Pasted image 20250601201000.png}
    \caption{Topologie du réseau}
\end{figure}

\section*{VLANs et Sous-réseaux DHCP}

\begin{longtable}{|c|c|c|c|c|}
\hline
\textbf{VLAN ID} & \textbf{Nom VLAN} & \textbf{Sous-réseau IP} & \textbf{Gateway} & \textbf{Plage DHCP} \\
\hline
10 & SALLE0 & 192.168.80.0/27 & 192.168.80.1 & 192.168.80.2 – 192.168.80.30 \\
\hline
20 & SALLE1 & 192.168.80.32/27 & 192.168.80.33 & 192.168.80.34 – 192.168.80.62 \\
\hline
30 & SALLE2 & 192.168.80.64/27 & 192.168.80.65 & 192.168.80.66 – 192.168.80.94 \\
\hline
40 & SERVEURS & 192.168.80.96/27 & 192.168.80.97 & Statique \\
\hline
\end{longtable}

\chapter{Configuration Réseau}
% Inclure ici les étapes de configuration si besoin, ou insérer d'autres images de captures.

\section*{Captures supplémentaires}
\begin{figure}[h!]
    \centering
    \includegraphics[width=0.8\textwidth]{captures/Pasted image 20250602070154.png}
    \caption{Exemple de configuration réseau}
\end{figure}

% Répétez ce bloc pour insérer d'autres images :
% \includegraphics[width=0.8\textwidth]{captures/NOM_DU_FICHIER.png}

\chapter*{Conclusion}
Ce projet nous a permis de mettre en œuvre les principes de sécurité et de communication dans un réseau intelligent à travers les outils de simulation et les protocoles standards du monde IoT.

\end{document}
